\section{Seconde formulation robuste}
\subsection{Modèle}
Le modèle robuste linéaire se base sur le pire des cas possible pour calculer un $epsilon$ optimal. On impose maintenant que
$$\sum_{i=1}^N \xi_i^2 \leq \gamma^2$$
%TODO Valeur de gamma
On s'attend donc à de meilleurs résultats car les zones de $\mathbb{R}_N$ où tous les $\xi_i$ ont une valeur absolue élevée sont supprimées.
On peut à nouveau reprendre les contraintes du primal et reformuler un problème d'optimisation, conique cette fois.
\begin{eqnarray}
\max_{\xi} dx^T \xi & \leq & \epsilon - d(\theta)^Tx \\
(t,\xi) & \in  & \mathbb{L}_{R^{n+1}}\\ 
t = \gamma
\end{eqnarray}
Il s'agit bien d'un problème d'optimisation conique. Son dual s'écrit :
\begin{eqnarray}
\min_{y} \gamma^2 y & \leq & \epsilon - d(\theta)^Tx \\
\begin{pmatrix}
y \\
d_1(\theta)x_1(\theta) \\
\vdots \\
d_n(\theta)x_n(\theta)
\end{pmatrix}
 & \in & \mathbb{L}_{R^{n+1}}
\end{eqnarray}
Par la dualité forte, l'objectif optimal du dual conique est supérieur à l'optimum du primal.\\
Celà nous permet encore une fois de réécrire notre problème de façon conique :
\begin{eqnarray*}
\min_{x,\epsilon ,y_1,y_2,y_3,y_4} \epsilon & & \\
y_{i}(\theta) & \geq & 0 \\
\begin{pmatrix}
y_i(\theta) \\
d_1(\theta)x_1(\theta)\\
\vdots \\
d_n(\theta)x_n(\theta)
\end{pmatrix}
& \succeq_{\mathbb{L}^{N+1}}& 0\\
& \forall i = 1,2,3,4 & 
\forall \theta \in S_e \cup P_e \\
\gamma^2 y_{1}(\theta) & \leq & \epsilon -d(\theta)^Tx \\
\gamma^2 y_{2}(\theta) & \leq & \epsilon + d(\theta)^Tx \\
\forall \theta \in S_e & & \\
\gamma^2 y_{3}(\theta) & \leq & \epsilon +1 -d(\theta)^Tx \\
\gamma^2 y_{4}(\theta) & \leq & \epsilon -1 + d(\theta)^Tx \\
\forall \theta \in P_e & &
\end{eqnarray*}
\subsection{Analyse des résultats}