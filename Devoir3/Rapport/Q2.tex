\section{Méthode de point intérieur}

Nous cherchons à définir une méthode à pas longs pour un problème d'optimisation du type suivant :
\begin{align*}
\min_{x_1,x_2} & c_1^Tx_1+c_2^Tx_2\\
\begin{pmatrix} A_1 & A_2 \end{pmatrix}
& = \begin{pmatrix} x_1 \\ x_2 \end{pmatrix} \\
x_1 & \geq 0 \\
x_2 & \in \mathbb{L}^{n_2}
\end{align*}
A cet effet nous supprimons les contraintes d'inégalité à l'aide de fonctions barrières. Le problème perturbé s'écrit donc
\begin{align*}
\min_{x_1,x_2}  & c_1^Tx_1+c_2^Tx_2 - \mu_k(\sum_{x_1i} \log(x_i)+ \log(\tau^2 - ||x_2||_2^2)) \\
\begin{pmatrix} A_1 & A_2 \end{pmatrix}
& = \begin{pmatrix} x_1 \\ x_2 \end{pmatrix} \\
\end{align*}
Comme le problème est convexe, nous pouvons écrire les conditions d'optimalité comme suit :
\begin{align*}
\bigtriangledown f & =  0 \\
\text{avec }f(x_1,x_2,z) & = 
c_1^Tx_1+c_2^Tx_2 - \mu_k(\sum_{x_1i} \log(x_i)+ \log(\tau^2 - ||x_2||_2^2)) -z^T(A_1x_1+A_2x_2-b)\\
& \Rightarrow & \\
\bigtriangledown_{x_1}f & =  c_1^T -s_1^T -z^TA_1 = 0 \\
\bigtriangledown_{x_2}f & =  c_2^T -s_2^T - z^TA_2 = 0 \\
\bigtriangledown_{z}f & =  -x_1^TA_1^T-x_2^TA_2^T+b^T = 0 \\
s_{1i} x_{1i} & = \mu_k \\
s_{2i} \frac{\tau^2 - ||x_{2}||_2^2}{2x_{2i}} & = \mu_k
\end{align*}
Où $z$ est le vecteur des multiplicateurs de Lagrange. On doit donc résoudre $F=0 $ avec 
$$
F : \begin{pmatrix}
x_1 \\ x_2 \\ y \\ s_1 \\ s_2 \\ t
\end{pmatrix}
\rightarrow
\begin{pmatrix}
A_1x_1+A_2x_2-b \\
A^T_1y + s_1 -c_1 \\
A_2^Ty + s_2 - c_2 \\
X_1S_1e - \mu_k e \\
TS_2e - \mu_ke \\
2X_2Te - (\tau^2-||x_2||_2^2)e
\end{pmatrix}
$$
On peut calculer la jacobienne : 
$$J_F = \begin{pmatrix}
A_1 & A_2 & 0 & 0 & 0 & 0 \\
0 & 0 & A_1^T & I & 0 & 0 \\
0 & 0 & A_2^T & 0 & I & 0 \\
S_1 & 0 & 0 & X_1 & 0 & 0 \\
0 & 0 & 0 & 0 & T & S_2 \\
0 & 2T + 2ex_2^T & 0 & 0 & 0 & 2X_2 

\end{pmatrix}$$
On a alors le système à résoudre pour obtenir le pas de Newton

\begin{align}
J_F \begin{pmatrix}
\Delta x_1 \\
\Delta x_2 \\
\Delta y \\
\Delta s_1\\
\Delta s_2\\
\Delta t\\
\end{pmatrix} & = & \begin{pmatrix}
0 \\
0\\
0\\
-X_1S_1e+\mu_k e \\
-TS_2e + \mu_k e\\
-2X_2Te+ (\tau^2-||x_2||_2^2)e\\
\end{pmatrix}
\label{eq:pasN}
\end{align}
On a alors l'algorithme suivant
\begin{algorithm}[!h]
\KwData{$A_1,A_2,c_1,c_2,b,\tau_L,\mu_0, \epsilon,\sigma,\nu$}
$(x_1^{0},x_2^{0},y^{0},s_1^{0},s_2^{0},t^{0})$ est tel que 
$\delta(x_1^{0},x_2^{0},y^{0},s_1^{0},s_2^{0},t^{0}) < \nu$
\For{$k = 0,1,2,... $, }
Calculer le pas de Newton $(\Delta x_1^{k},\Delta x_2^{k},\Delta y^{k},\Delta s_1^{k},\Delta s_2^{k},\Delta t^{k})$ à l'aide du système (\ref{eq:pasN})\;
Poser $x^{k+1} = x^k + \alpha_k \Delta x^k$
%TODO : choix du alpha
\end{algorithm}