\section{Méthode de point intérieur}

Nous cherchons à définir une méthode à pas longs pour un problème d'optimisation du type de \eqref{eq:SystToSolve}. Nous supprimons les contraintes d'inégalité à l'aide de fonctions barrières. Le problème perturbé s'écrit donc
\begin{align*}
\max_{x}  & \frac{1}{\mu_k}c^Tx - \varphi(x) \\
Ax & =  b
\end{align*}
Avec $\varphi(x) = \sum_{i} \log(4k^2 - (x_i-x_{i+nn})^2+(2x_{i+2*nn})^2)$.

On résout ce problème à l'aide d'une méthode de point intérieur à pas longs.
On a alors l'algorithme suivant
\begin{algorithm}[!h]
\KwData{$A,c,b,x_0,y_0,\tau,\mu_0, \epsilon,\sigma,\nu$}

\While{$k <itMax$, $\mu_k>\mu_f$ }{
$mu_k = \sigma mu_{k-1}$\;
Appliquer Newton jusqu'à ce que $\delta(x,\mu_k)<\tau$\;
}
\end{algorithm}


Comme le problème est concave, on écrit $G(x,y) = \frac{1}{\mu_k}c^Tx -\varphi(x) +y^T(Ax-b)$; et les conditions d'optimalité sont $\nabla G=0 $, ou plus explicitement :
\begin{align*}
\nabla_{x}G & =  \frac{1}{\mu_k}c -\nabla \varphi -A^Ty = 0 \\
\nabla_{y}G & =  Ax-b= 0 \\
\end{align*}
Où $y$ est le vecteur des multiplicateurs de Lagrange. Il s'agit d'un système non-linéaire d'équations ($\varphi$ non-linéaire) qu'on peut résoudre par la méthode de Newton. 
Le pas de Newton pour cette équation est donné par la résolution du système 
\begin{equation}
\nabla^2G  \begin{pmatrix}
\Delta x \\ \Delta y
\end{pmatrix}
= 
-\nabla G
\Longleftrightarrow
\begin{pmatrix}
-\nabla^2 \varphi(x) & A^T \\
A & 0
\end{pmatrix} \begin{pmatrix}
\Delta x \\ \Delta y
\end{pmatrix} = \begin{pmatrix}
-(\frac{1}{\mu_k}-\nabla \varphi + A^Ty)\\
0
\end{pmatrix}
\end{equation}

Nous avons donc opté pour la méthode qui consiste à modifier le pas de Newton pour y intégrer les contraintes d'égalité.
Pour ce qui est de l'itéré initial $(x_0,y_0)$, on remarque que $(0,y_0)$ est une solution admissible pour tout $y_0$.